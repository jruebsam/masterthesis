\chapter*{Introduction}
\addcontentsline{toc}{chapter}{Introduction}

Fluid flows play an important role in the study of geophysical systems.
Of particular interest are the flows on large scales in the ocean and the atmosphere.
Furthermore rotating and convection driven flows in the core of the earth, are a potential mechanism
for magnetic field generation and current interest of research.

The flow in a rotating fluids exhibits characteristic properties as a result of the coriolis force,
physical solutions are of the form of inertial waves.
Numerical \citep{Sauret2012}, \citep{Duguet} and experimental
studies \citep{QUOTE} of inertial waves excitation in rotating cylinders have been performed,
inertial oscillations in spherical shells were investigated for example by \citep{Tilgner1999}.
%A further extension of the rotating fluids system by a coupling to magnetic fields
%The study of geo dynamos is possible by an extension of
%the rotating fluids system with a coupling to magnetic fields
The study of dynamos, i.e. the magnetic field of the earth, is obtained by additionally
considering an electrically conducting fluid.
Numerical simulations of precession driven dynamos \citep{Tilgner2005}
and convection driven dynamos \citep{Tilgner2012} were performed, furthermore
an on going experimental research project is the DRESDYN dynamo \citep{Stefani2015}.

In the fluid dynamics research group of the institute for geophysics
a recent development is the numerical simulation of such fluid flows using GPU optimized algorithms.
The use of GPUs for scientific computing has become increasingly popular in the last years.
A vast number of applications  can obtain a massive speed-up and
high oriented frameworks like the NVIDIA CUDA API simplify the creation
and maintenance of parallelized algorithms.
The current GPU implementation of the algorithm used in the research group is restricted
to simulations in cuboid domains.
Since an equidistant cartesian grid is used for the discretization of the fluid
domain it is not possible to perform simulations in complex-shaped geometries,
since the boundaries of the fluid domain  do not coincide with the grid points.

\bigbreak

The first goal of this master thesis is to extend the existing GPU algorithm
in order to enable the simulation of fluid flows in complex-shaped geometries.
The use of unstructured or body conforming grids is difficult to implement,
since on a GPU a regular and homogeneous memory access is required to enable a high memory bandwidth \citep{CUDABP}.
Hence, the objective is to use a set of methods which retain the speed of the original GPU algorithm and use a cartesaon grid.

This requirement is fulfilled by Immersed Boundary methods.
An overview of these methods is given in \citep{Mittal2005}, \citep{Gornak2013}.
The concept behind these methods is the embedding of the curved boundary into the cartesian grid
by using additional forcing terms and interpolation methods at the boundaries.
Certain Immersed Boundary methods use a continuous forcing approach where the boundarie is approximated by Lagrangian points,
and the forcing term acts over multiple grid points \citep{Mittal2005},
similar approach is given by the Volume-Penalization method \citep{Lulff2011}.
Direct forcing approaches exist where the boundary conditions are obtained by setting the
velocity values directly at the nearest points to the boundary \citep{Fadlun2000}.
A bilinear interpolated version of the Direct Forcing approach is used in \citep{Gornak2013},
which was furthermore extended to work on a GPU \citep{DeLeon2012}.
In this thesis different Immersed Boundary methods will be introduced, implemented and
validated by different test cases.

\bigbreak

The second goal of this master thesis is to perform a numerical study on a rotating system.
The system is given by a rotating cone whose rotation rate is modulated in order to excite inertial waves.
An experimental study of this system shows that the apex of the cone acts as an attractor, inertial modes do not exist.
By inserting a bottom plate into the cone an inertial wave can be reflected on the bottom of the cone \citep{Beardsley1970} and
inertial modes are present.
The objective is to find out if the implemented methods are able to reproduce experimental results from .


\bigbreak

The content of this thesis is seperated into the following chapters
\begin{description}
\item[Chapter 1] An introduction to the theoretical concepts is given.  This includes a description of the Navier-Stokes equations
                    and the properties of fluid flows in rotating systems.
\item[Chapter 2] This capter introduces the numerical methods, in particular the use of finite difference methods,
                    numerical stability criteria and the method of artifical incompressibility.
\item[Chapter 3] This chapter presenfts the implemention of a finitite difference  algorithm on a GPU

\item[Chapter 4] An introduction to  the Immersed Boundary methods used in this thesis will be given, followed by
                 the numerical validation with different test cases

\item[Chapter 5] The results of the simulations of a librating cylinder, a librating cone and frustum will be presented and discussed
\end{description}

%The first obective can  be described as fallow
%- combine the use of gpu with ibm to enable simulation of curved geometries on a gpu
%-  this thesis deals with the simulaiton of geophysical flows on  curved geometries using a gpu based algorithm (muss das an den anfang?)
%- in order to test this on a physicl system  librating cone
%- study the phyisical properties of this system inertial waves  and heliciy
%
%-this thesis is build by three esction
%-in the first part theorie etc
%- in the second part introduction of ibm and the validaion of ibm with stanrad test cases
%- in the third part a study of  librating cone with the use of immersed boundaryie methods will be performed












