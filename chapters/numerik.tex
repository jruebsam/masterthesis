\chapter{Numerical Methods}

\section{Introduction}

This chapter focuses on the methods that are used for the numerical computations in this thesis.
In order to compute the temporal evolvement of a fluid system from its initial state, it is necessary to discretize
the equations of motion by using different numerical schemes.\\
For this purpose various discretization approaches, for example finite-element and finite-volume methods, exists.
Here we will introduce the method of finite-difference stencils for the spatial and a third order Runge-Kutta method for the discretization in time.
Furthermore we will introduce the method of artificial compressibility, which can be used to avoid the numerical expensive solution of a poisson equation.
The choice of these methods in combination with the usage of cartesian grids is in particular time saving when performing computations on the gpu, as we
will see in chapter \ref{chapter:cuda}.

\section{Finite Differencing Schemes}

We start with a brief introduction of finite difference methods.
The interested reader is referred to \citep{ferziger99} for a more general overview, from where this section is adapted.
The partial differential equations we want to solve in this thesis are of the form,

\begin{align}
    \label{numerik:pde_allg}
    \pdn[\Phi]{t} = \left(\sum_{x_i=x,y,z}\left( A_i \pdn[]{x_i}  + B_i \pdn[^2]{x_i^2}\right) + C +  \vec{u}\vec{\nabla} \right) \Phi = \mathcal{L} \Phi
\end{align}
    %\pdn[\Phi]{t} = A \pdn[^2\Phi]{x^2}  + B \pdn[^2\Phi]{x^2}     + C(\vec{r}, \vec{u}, t) +  \vec{u}\left(\pdn[^2\Phi]{x^2} +  \pdn[^2\Phi]{x^2} + \pdn[\Phi]{x}\right) = \mathcal{L}

where $\Phi(\vec{r}, t)\in\mathbb{R}$ and $\mathcal{L}$ is a differential operator, containing spatial derivatives up to second order.
The numerical integration can be divided into two steps, the calculation of $\mathcal{L}$, which we want to discuss in this
section and secondly the integration in time.
The exact calculation of the spatial derivatives in $\mathcal{L}$ is numerically not possible.\\
Due to the limited storage capacity and computation time of computers,
it is necessary to discretize the domain, on which the PDE should be solved and find a adequate approximation of these operators.
Here we will fall back to the  one-dimensional case, the implementation for three dimensions will be discussed in chapter \ref{chapter:cuda}.\\

Let $\Omega = \{x \in \mathbb{R} \;|\; 0 \leq x \leq L\}$ be the domain on which we want to solve equations of the type \ref{numerik:pde_allg}.
For the discretization we divide $\Omega$ into $N$ equidistant points $x_i = \sum_i \Delta x_i$, with the position index ${i\in\{[0, N-1]|i\in\mathbb{N}\}}$
and $\Delta x_i = x_{i+1} - x_i = L/(N-1)$.
\footnote{EVTL?}
We assume that $\Phi$ is a  continuous differentiable function.
Local to a grid point $x_i$ and $\Phi$ can than be expressed with a Taylor series [CITE].

\begin{align}
    \label{num:taylor}
    \Phi(x) = \sum_{n=0}^{\infty} \pdn[^n\Phi(x_i)]{x^n} \frac{(x - x_i)^n}{n!}
\end{align}

By evaluating the Taylor expansion at different points, we obtain expressions for the first derivative. For example
a combined evaluation at the points $x_{i+1}$, $x_{i-1}$ leads to the expression

\begin{align}
    \label{num:cds}
    \left.\left(\pdn[\Phi]{x}\right)\right|_{i} = \frac{\Phi_{i+1} - \Phi_{i-1}}{x_{i+1} - x_{i-1}}
     - \frac{(x_{i+1} - x_i)^2 - (x_i - x_{i-1})^2}{2 (x_{i+1} - x_{i-1})}\left(\pdn[^2\Phi]{x^2}\right)_i + \mathcal{O}(\Delta x^3)
\end{align}

For a constant grid size, that is $\Delta x := \Delta x_i = \text{const.}$, the second order term in equation \ref{num:cds} vanishes.
By neglecting all terms of higher order, we obtain a approximation for $\partial_x \Phi$ of second order.
This is the so-called central-difference (CDS) scheme. A single point evaluation of \ref{num:taylor} at $x_{i+1}$ and $x_{i-1}$, results in the forward- (FDS) and
backward- (BDS) scheme of first order. A comparison of the FD-schemes is given in table \ref{num:df_table}


\bgroup\large
\begin{table}[!tbp]
\centering
\def\arraystretch{2.2}%
\begin{tabular}{c c c c c}\toprule
Scheme-Name & Stencil & Truncation Error & Evaluation at\\[0.5ex]
\midrule
Forward  (FDS) & $\left(\pdn[\Phi]{x}\right)_i =  \frac{f_{i+1} - f{i}}   {\Delta x}$ & $\mathcal{O}(\Delta x)$  &$x_{i+1}$\\
Backward (BDS) & $\left(\pdn[\Phi]{x}\right)_i = \frac{f_{i}    - f_{i-1}}{\Delta x}$  &$ \mathcal{O}(\Delta x)$ & $x_{i-1}$\\\
Central  (CDS) & $\left(\pdn[\Phi]{x}\right)_i = \frac{f_{i+1}  - f_{i-1}}{2\Delta x}$ &$ \mathcal{O}(\Delta x^3)$& $x_{i+1}$ \& $x_{i-1}$\\
\\
\bottomrule
\label{num:df_table}
\end{tabular}
\caption{Different FD-Schemes}
\end{table}
\egroup

\begin{figure}[!btp]
  \centering
    \resizebox{0.9\textwidth}{!}{
   \import{gfx/numerik/}{finite_differenzen.pdf_tex}
  }
  \caption{Approximation of the function $\Phi$ by different finite difference schemes.}
  \label{num:fd_image}
\end{figure}

The numerical error which is made by neglecting the higher order terms, is in general referred to as the truncation error of a FD-scheme.
It should be noted that the number of grid points used for the approximation, does significantly affect the resulting error.\\
Finally figure \ref{num:fd_image} shows a visual comparison for the three different stencils.
This example once again illustrates that when approximating a function, which contains higher order terms i.e. at a local maximum,
the  CDS-scheme gives better results.\\
For the computation of the second derivative, one approach is to evaluate equation \ref{num:taylor} halfway between two points at the positions $x_{i\pm\frac{1}{2}}$

\begin{align}
    \left.\left(\pdn[^2\Phi]{x^2}\right)\right|_{x_i} =
     \frac{\left.\left(\pdn[\Phi]{x}\right)\right|_{i+\frac{1}{2}}-
     \left.\left(\pdn[\Phi]{x}\right)\right|_{i-\frac{1}{2}}}
    {\frac{1}{2}(x_{i+1} - x_{i-1})} + \mathcal{O}(\Delta ^2)  \approx
    \frac{\Phi_{i+1} - 2\Phi_i + \Phi_{i-1}}{\Delta x^2}
\end{align}

For the approximation of the first derivatives, it is necessary to use the FDS-scheme at $x_{i+1/2}$ and the BDS-scheme at $x_{i-1/2}$ to obtain
a second order accuracy.\\
So far we introduced methods up to an accuracy of second order. The fourth order methods which are used in this thesis are given by

\begin{align}
    \left(\pdn[\Phi]{x}\right)_i &\approx \frac{-\Phi_{i+2} + 8\Phi_{i+1} - 8\Phi_{i-1} + \Phi_{i-2}}{12\Delta x} \\
    \left(\pdn[^2\Phi]{x^2}\right)_i &\approx \frac{-\Phi_{i+2} + 16\Phi_{i+1} -30\Phi_i + 16\Phi_{i-1} - \Phi_{i-2}}{12\Delta x^2} \\
\end{align}

The derivation of these equations can be performed in analogy to the second order schemes, but by using a five-point
stencil. The interested reader is referred to \citep{Fornberg1988}.

\newpage

\section{Runge-Kutta Method}

With the finite difference approximation $\mathcal{L^*}$ of the differential operator $\mathcal{L}$,
we are able to solve equation \ref{numerik:pde_allg} by separation of variables.

\begin{align}
    \Phi(t) = \Phi(0) + \int_{t_0}^{t} \mathcal{L^*} (t, \Phi(t)) \dif t
\end{align}

For a numerical computation, the integration interval $[t_0, t]$  is splitted into sub-intervals $[t^n, t^{n+1}]$ of length $\Delta t$ and integrated piecewiese.
In order to approximate the integration on a sub-intervall, we know want to introduce the use of Runge-Kutta methods based on \citep{Sarbach2012}
The idea behind this method is to iteratively evaluate $\mathcal{L^*}$ at $s$ different times $\tau_i=c_i \Delta t$ with $c_i \in \{\mathbb{R} | c_i<1  \}$,
that is
\begin{align}
 k_i = \mathcal{L^*} \left(\tau_i, \Phi(t_n) + \Delta t \left(a_{21} k_{s-1} + \dots + a_{s, s-1} k_{s-1} \right)\right)
\end{align}

By using a coefficient weighted average of the $k_i$, we obtain

\begin{align}
    \Phi(t^{n+1}) = \Phi(t) + \Delta t \left( \sum_{i=1}^s b_i k_i \right)
\end{align}

The accuracy of an s-stage runge-kutta method depends of the right choice of coefficients.
In this thesis we use a third-order accurate scheme.
The coefficients are given by the butcher tableau \citep{QUELLE}
\footnote{oder quelle}

\begin{align}
    \label{num:butcher}
    \def\arraystretch{1.5}%
    \begin{array}{c|c}
        \mathbf{c} & A\\
        \hline     & \mathbf{b^T} \\
    \end{array}
    \qquad &= \qquad
    \def\arraystretch{1.5}%
    \begin{array}{c|ccc}
            0 \\
                    \frac{1}{3} & \frac{1}{3} \\
            \frac{3}{4} & \frac{-3}{16} & \frac{15}{16} \\
            \hline & \frac{1}{6} & \frac{3}{10} & \frac{8}{15}
    \end{array}
\end{align}

For the implementation of this method, we have to concern, that we have to deal with a large number of variables.
In a three dimensional fluid domain the number of grid points scales with a power of three, with an increase in the resolution.
Therefore we want to introduce a low-storage scheme of method \ref{num:butcher},  first introduced by \citep{Williamson1980}.
By keeping the information from previous computation steps, the scheme can than be translated into an iterative algorithm,
where the required amount of storage can be reduced to two registers $\Phi$ and $Q$.

\begin{align}
    \begin{split}
    Q_1 = \Delta t \mathcal{L}^*\left(\Phi^n\right)\qquad &\Rightarrow \qquad \Phi^{1} = \Phi^n + \frac{1}{3}Q_1 \\
    \Rightarrow Q_2 = \Delta t \mathcal{L}^*\left(\Phi^1\right) - \frac{5}{9} Q_1 \qquad &\Rightarrow \qquad \Phi^{2} = \Phi^1 + \frac{15}{16}Q_2 \\
   \Rightarrow  Q_3 = \Delta t \mathcal{L}^*\left(\Phi^2\right) - \frac{153}{128} Q_2 \qquad &\Rightarrow \qquad \Phi^{n+1} = \Phi^2 + \frac{8}{15}Q_3 \\
    \end{split}
\end{align}


\section{Numerical Stability}

It is important to consider the stability of a numerical method. In general it is said that a method is numerical stable if the error
introduced by approximations and roundoff errors does not grow over time \citep{ferziger99}.\\

\subsection{Runge-Kutta Scheme}

Suppose we have stationary solution for equation \ref{numerik:pde_allg}.
The system should remain unaffected by a further integration, even with the introduction of small numerical errors.
In order to obtain a stability condition it is sufficient enough to perform a linear stability analysis.
The most common approach is to study the one-dimensional linearized and diagonalized test problem (\citep{BLABLA})

\begin{align}
\label{num:rk_stab}
\frac{\dif y}{\dif t} = \lambda \Phi
\end{align}

with the eigenvalue $\lambda \in \mathbb{C}$ of a linearized operator.
It can be shown that the discretization of equation \ref{num:rk_stab}, using a Runge-Kutta scheme of $s$-order,
results in  the mapping (\citep{BLABLA})

\begin{align}
    y_{i+1}  = \left(\sum_{p=1}^s \frac{(\Delta t \lambda)^p}{p!}  + 1 \right) y_i = f(\Delta t\lambda)y_i
\end{align}

\begin{figure}[!tp]
  \begin{minipage}[c]{0.6\textwidth}
      \includegraphics{gfx/numerik/rk_stability.pdf}
  \end{minipage}\hfill
  \begin{minipage}[c]{0.3\textwidth}
  \caption{Stability regions for $\Omega_s$ for different Runge-Kutta methods}
  \label{fig:num_rkstab}
  \end{minipage}
\end{figure}

The discretized system contains a stable fixed point when the condition $|f|\leq 1$ is satisfied.
Figure \ref{fig:num_rkstab} shows the different stability regions for Runge-Kutta schemes up to fourth order.
The calculation can be found in (APPENDIX)
For an $s$ order scheme the region of absolute stability is independent of the used implementation \citep{canuto2007}.
\footnote{For an $s$-th order RK-scheme different coeffiicent $a_ij$, $b_j$ and $c_i$ can be used.}
It can be noted that the stability region $\Omega_s$ increases with the uses of higher order schemes.
The substantial detail is that for method of third and fourth order, the imaginary axis lies within $\Omega_s$.
Thus these methods should be preferred for  CFD-simulations, since they tend to stabilize numerical oscillations \citep{DIPLOM}.

\subsection{Von-Neumann Error Analysis}

-idee fourier mode
-nur eine warum

\begin{align}
    \epsilon = \sum e^{at}e^{i k_m x}
\end{align}

    single mode of the form $e^{at} e^{i k_m x}$


diffusion equation $\partial_t \Phi = \alpha \Delta \Phi$
we get

\begin{align}
    \frac{\Phi^{n+1} - \Phi^n}{\Delta t} = \frac{\alpha}{\Delta x^2}\left(\Phi_{i+1}^n - 2\Phi_{i}^n + \Phi_{i-1}^n\right)
\end{align}

resulting in

\begin{align}
    e^{\alpha \Delta t} = 1 - \frac{\alpha}{\Delta x^2} \sin^2\left(\frac{k_m \Delta x}{2}\right)
\end{align}

the amplification $G := \epsilon_j^{n+1} / \epsilon_j^n  = e^{\alpha \Delta t }$
stable for $|G|<1$ results in .

\begin{align}
    C := \frac{\alpha \Delta t}{\Delta x^2} \leq \frac{1}{2}
\end{align}

the left side of equation () is also known as courant number blabla.
\begin{align}
    C := \frac{c \Delta t}{\Delta x} \leq \frac{1}{2}
\end{align}

until know single differential operators, however when evaluating equations of the form, where an advection term occurs with an diffusion term
we cet the condition

\begin{align}
    Pe := \frac{u \Delta x}{\alpha} < 2
\end{align}

The leftside term is also denoted as Peclet number  and
when diffusive fluxes to small oscillations blabla
The Upwinding Scheme first and third  order





\newpage

\section{Method of Artificial Kompressibility}

Until now we have ignored the fact that the equations of motion (), cannot directly be solved for the pressure.
To obtain an equation for $p$, we can take the divergence of equation (), and apply the continuity equation.
The resulting equation for the pressure reads

\begin{align}
    \Delta p =  -\nabla \left( \sum \vec{F}_{\text{ext.}} - (\vec{u} \vec{\nabla}) \vec{u}\right)
\end{align}

-laplace gleichung für den druck schwer zu lösen
-instead we introduce

\begin{align}
    \underbrace{\pdn[\rho]{t} +  \nabla \vec{v} = 0}_{\label{bla}} \qquad , \qquad p =\rho/\delta \qquad, \qquad c=\frac{1}{\delta^{1/2}}
\end{align}

where $c$ and so on

\begin{align}
    \label{numerik:nsac}
    \pdn[u]{t} + \left(\vec{u} \vec{\nabla}\right) \vec{u} &= -c^2\nabla\rho + \Rey \Delta \vec{u}+\sum \vec{F}_{\text{ext.}}
\end{align}

-machnumber
\begin{align}
    M = \frac{1}{c}\max\left(\sum_i |\vec{u_i}]^2\right)^{1/2}
\end{align}


-stiff problem
-implizit methods
-problem not suited for gpus
\section{Numerical Viscosity}

\begin{align}
 \label{NUMERIC:NUMVIS}
\end{align}

