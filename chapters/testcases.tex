\section{Validation}

%This part of the thesis the thesis deals with the numerical validation of the immersed
%boundary methods
%In the first part of the chapter an introduction to different test cases
%will be given. Three different test problems were chosen.
%The flow between to planes, also known as Planar Poiseuille flow, the flow in a Pipe, referred to as Hagen-Poiseuille flow and the flow
%between two rotating cylinders, know as the Taylor-Couette system.
%In the second part of the chapter the result for the different test cases will be presented and discussed.
%In the first part of the chapter an introduction to different test cases
%will be given.
This section presents the numerical validation of the Immersed Boundary methods.
The objective of the validation is to determine the numerical accuracy and
the numerical stability for each method.
Furthermore, it is important to test if conservation laws, in particular the conservation of mass,
are fulfilled.
For the validation three different test problems were chosen.
The flow between two parallel planes which is also known as planar Poiseuille flow, the flow in a Pipe which is referred to as Hagen-Poiseuille flow and the flow
between two rotating cylinders which is known as the Taylor-Couette system.

%\paragraph{Conventions}\mbox{}\\
\subsection{Conventions}

In this section the following abbreviations will be used

\begin{multicols}{2}
\begin{description}
    \item[VP]{Volume-Penalization Method}
    \item[DF]{Direct Forcing Method}
    \item[IP]{Interpolation Method}
    \item[VF]{Volume-Fraction Method}
    \item[FD2]{Finite Difference Schemes of second order}
    \item[FD4]{Finite Difference Schemes of fourth order}
\end{description}
\end{multicols}

For example, DF-VF FD2 method, refers to the Direct-Forcing method, extended with the Volume-Fraction
method and the use of second order finite difference schemes.

%\paragraph{Grid Convergence Studys}\mbox{}\\
\subsection{Grid Convergence Studies}

For the validation multiple grid convergence studys were performed.
The concept of a grid convergence study is to vary the grid resolution of the numerical domain and
calculate the error of the simulation with respect to an assumed theoretical solution.
The relative error $\epsilon$ is computed by the $l_2$-norm by

\begin{align}
    \epsilon = \frac{\int \dif V \left(\Phi^{\text{th}} - \Phi^{\text{num}}\right)^2}{\int \dif V \left( \Phi^{\text{th}} \right)^2}
     = \frac{\sum_{i,j,k}^{N_x, N_y, N_z}
      \left(\Phi^{\text{th}}_{i,j,k}  - \Phi^{\text{num}}_{i, j, k}  \right)^2}
     {\sum_{i,j,k}^{N_x, N_y, N_z} \left( \Phi^{\text{th}}_{i,j,k} \right)^2}
 \end{align}

where $\Phi^{\text{th}}$ corresponds to the theoretical and $\Phi^{\text{num}}$ to the numerical solution of any field variable $\Phi$.
Often the numerical error is given by a power law of the form $\epsilon = N^\lambda$, i.e. for a finite difference scheme of second order
this is the truncation error given by the remaining terms  of the Taylor expansion.
For an evaluation of the error convergence the results of the grid convergence study are, if possible, fitted linearly in the log-log space, to obtain the
convergence rate $\lambda$.
The following testcases are used for a numerical validation.

%In general it is necessary to obtain a good evaluation of the numerical truncation error,
% the numerical stability over longer periods of time
%and Grid convergence studys against theoretical and high-resolution numerical solutions
%will be performed and compared for the different IBMs.
%In order to ensure a correct numerical behavior of the introduced methods, a
%Multiple examples from simple to more complex test cases are introduced in this section.

\subsection{Laminar Poiseuille Flow}
\label{vali:sec_lpflow_setup}

\begin{figure}[!bp]
  \begin{minipage}[c]{0.6\textwidth}
      \centering
        \resizebox{0.9 \textwidth}{!}{
       \import{gfx/immersed_boundary/poiseuille_flow//}{setup.pdf_tex}
      }
  \end{minipage}
  \begin{minipage}[c]{0.3\textwidth}
      \caption{Flow in a channel of height $\Delta h = h_2 - h_1$, with periodic boundaries in $x$ and $y$ direction and
       a predefined pressure gradient $\nabla p$.
      \label{validation:setup_pf}
      }
  \end{minipage}
\end{figure}

The numerical setup of this test case is presented in Fig. \ref{validation:setup_pf}.
It consist of two infinite extending  planes at $z=h_1$ and $z=h_2$, which are oriented
parallel to the xy-plane, with a distance $\Delta h = h_2 - h_1$.
An infinite long channel is numerically realized by using periodic boundaries in x- and y-direction.
For the walls immersed boundaries, or in case of the default algorithm No-Slip boundaries, are used.
The fluid flow in the channel is a result from a predefined constant pressure gradient $\nicefrac{\partial p}{\partial x}$
in the fluid domain.
In the implementation this pressure gradient is added into the Navier-Stokes equation as an additional forcing term.
For the steady state the flow is independent of the $x$ and $y$ coordinate.
According to \citep{Kundu2012}, the equations of motion, here in the non-dimensionalized form, are given by

\begin{align}
    \label{vali:pflow_navstok}
    \frac{\partial v_x}{\partial t} &= - \frac{\partial p}{\partial x}
     + \frac{1}{Re} \frac{\partial^2 v_x}{\partial z^2} = 0,
\end{align}

where $Re = \nicefrac{V_{m}\Delta h}{\nu}$, with the non-dimensionalization defined as

\begin{align}
    \text{Length:}\qquad &  \vec{r} = \frac{\vec{r^*}}{\Delta h}  &
    \qquad \text{Velocity:}\qquad& \vec{v} =  \frac{\vec{v^*}}{V_{\text{M}}}\\
    \text{Time:}  \qquad & t = \frac{t^* \cdot V_\text{M}}{\Delta h}&
    \qquad  \text{Pressure:}\qquad & p = \frac{\nabla p^*}{\rho V_\text{M}^2}
\end{align}
%For this system the non-dimenzionalization is given by the scales
%    $r^* = \nicefrac{r}{\Delta h}$, $v^*=\nicefrac{V_(\text{m})}{\Delta h}$,
%    $t^* = \nicefrac{V_{\text{m}}}{\Delta h}$ and $p^* = p \rho V_{\text{max}}^2$.

where $V_\text{M}$ is defined by the maximal velocity in the channel $\max(v_x(z))$.
By the integration of Eq. \ref{vali:pflow_navstok} it follows that
\begin{align}
v_x &= \frac{1}{2}\frac{\partial p}{\partial x}\Rey z^2 + zc_1 + c_2.
\end{align}

The integration constants are obtained by setting $v_x(h_1) = v_x(h_2) = 0$:
\begin{align}
c_1 = A\frac{h_1^2 -h_2^2}{h_2 - h_1} = -A(h_1+h_2)\qquad ,& \qquad
c_2 = A(h_1(h_1 + h_2) - h_1^2) = Ah_1h_2,
\end{align}

with the definition $A:=\frac{1}{2}\frac{\partial p}{\partial x} Re$.
The velocity is then given by the quadratic function
\begin{align}
\label{vali:pflow_theosol}
v_x &= A(z^2 - z(h_1 + h_2) + h_1h_2).
\end{align}

The maximum velocity and position are given by
\begin{align}
z_{\text{M}} &= \frac{h_1+h_2}{2} \wedge v_{\text{M}} = A\left(h_1h_2 - \frac{(h_1 + h_2)^2}{4}\right).
\end{align}

Due to the non-dimensionalization it is necessary that $V_{\text{M}} \overset{!}{=}  1$.
It follows that

\begin{align}
    \label{vali:pflow_pcondi}
\frac{\partial p}{\partial x} &= \frac{2}{Re}\frac{1}{\left(h_1h_2 - \frac{(h_1+h_2)^2}{4} \right)}
\end{align}

has to be fullfilled as a necessary condition for the pressure gradient.
The Poiseuille flow is used in particular as a first validation test case
for the VP, DF method  and the basic implementation.

%\paragraph{Test of the Basic Algorithm}\mbox{}\\
\subsubsection{Simulation: Test of the Basic Algorithm}

The purpose of the first simulation is to test  the basic
implementation of the algorithm which does not use Immersed Boundary methods.
For the Poiseuille flow this is still possible since the geometry is non-curved
and parallel to the cartesian grid. The heights for this setup are set to $h_1=0$ and $h_2=1$.
A grid convergence test was performed with the main simulation parameters given by
\footnote{For reasons of clarity only the important simulation parameters, depending on the setup and the performed simulation,  will be presented in this thesis.}
\footnote{The size in $l_x/l_y$ direction is set to the smallest possible value which is 8 grid points due to the GPU algorithm which uses a blocksize of 8x8.}

\begin{center}
\vspace*{0.7ex}
\begin{tabular}{c|c|c|c|c|c|c }
 $ N_z  $                   & $\Delta t$ & $\Delta z$            & $\Rey$  & $c^2$   & $l_z, l_x/l_y$ & $T_{end}$\\
\hline
 $[8, 256], \Delta N = 8 $& $10^{-4}$ & $\nicefrac{1}{N - 1}$ & 500     & $500$   &  1, 8$\Delta z$ &  10\\
\end{tabular}.
\vspace*{0.7ex}
\end{center}

The simulation was performed for FD2 and FD4 methods.

\subsubsection{Simulation: Test of the Volume-Penalization Method}
%\paragraph{Test of the Volume-Penalization Method}\mbox{}\\

%With the given theoretical solution, the next objective is the comparison
%to the default implementation, the volume penalization method and the direct forcing method.
%Since we have a flow parallel to the grid  it does not make sence to compare it to the interpolation methods
%For the comparision with a theoretical solution it is necessary to ensure that
% the surface grid points match with the total height $h$ of the channel.

For the Immersed Boundary methods the upper and lower boundaries of the channel are realized by the masking function
\begin{align}
H(z) = \begin{cases}
                    0, & \text{for \  }  h_1 \leq z \leq h_2 \\
                    1, & \text{else}.
             \end{cases}
\end{align}.

For the VP-method an additional forcing term for Eq. \ref{vali:pflow_navstok} of the form

\begin{align}
    \vec{f} = -\frac{H}{J}\vec{v}
\end{align}

has to be introduced (see Sec. \ref{chap:ibm_volpen}).
Where the non-dimensionalized damping coefficient is given by $J = \nicefrac{V_{m}}{\eta}$.
A first test is an investigation of the error of the velocity profile, with a variation of the Reynolds number and the damping coefficient $J$.
To ensure that the channel width is equal to $\Delta h = 1$, the total height of the simulation domain was set to $l_z\approx2.01587$,
with $h_1=0.5$ and $h_2=1.5$.
This ensures that the grid points overlap exactly with the masking function at $h_1$ and $h_2$.
A simulation series was performed for the VP-FD2 method with the simulation parameters

\begin{center}
\vspace*{0.5ex}
\begin{tabular}{c|c|c|c|c|c|c|c }
 $ \Rey  $                      & $J$ & $N_z$ &  $\Delta t$ & $\Delta z$            & $c^2$   & $l_z, l_x/l_y$ & $T_{end}$\\
\hline
 $[100, 500], \Delta \Rey = 25 $& $[10^{-5}, 5\cdot10^{-1}]  $ & 64 &  $10^{-4}$ & $\nicefrac{1}{N - 1}$   & $500$   & $\approx{2.015871}, 8\Delta z$ & 10\\
\end{tabular}.
\vspace*{0.5ex}
\end{center}

The stepping of $J$ was varied such that each order of magnitude is covered by two values, i.e. $10^{-1}$ and $5\cdot10^{-1}$.
A value of $J=10^{-5}$ corresponds to a relative strong damping force in comparison to a weak damping at a value of $J=10^{-1}$.
As a second test a grid convergence study was carried out with a constant Reynolds number,
where the resolution was varied between $N_z\in [8, 280]$ with $\Delta N_z = 8$. VP-FD2 and VP-FD4 methods were tested with the parameters

\begin{center}
\vspace*{0.5ex}
\begin{tabular}{c|c|c|c|c|c|c|c }
 $ N_z  $                      & $J$ &  $\Delta t$ & $\Delta z$            & $\Rey$  & $c^2$   & $l_z, l_x/l_y$ & $T_{end}$\\
\hline
 $[8, 280], \Delta N_z = 4 $& $[10^{-4}, 5\cdot10^{-2}]  $ &  $10^{-4}$ & $\nicefrac{1}{N - 1}$ & 500     & $500$   & $\approx{2.015871}, 8\Delta z$  & 10\\
\end{tabular}.
\vspace*{0.5ex}
\end{center}

\subsubsection{Simulation: Test of the Direct-Forcing Method}
%\paragraph{Test of the Direct-Forcing Method}\mbox{}\\

Since the DF method does not depend on a damping coefficient (see Sec. \ref{chap:ibm_dirforce}),
it is sufficient enough to carry out a grid convergence study.  The simulation parameters are given by

\begin{center}
\vspace*{0.3ex}
\begin{tabular}{c|c|c|c|c|c|c }
 $ N_z  $                       &  $\Delta t$ & $\Delta z$            & $\Rey$  & $c^2$   & $l_z, l_x/l_y$ & $T_{end}$\\
\hline
 $[8, 280], \Delta N_z = 4 $ &  $10^{-4}$ & $\nicefrac{1}{N - 1}$ & 500     & $500$   & $\approx{2.015871}, 8\Delta z$  & 10\\
\end{tabular}.
\vspace*{0.3ex}
\end{center}
The simulation was performed for FD2 and FD4 methods.

\subsection{Hagen-Poiseuille Flow}

\label{vali:section_hpflow_start}

In the previous test case the channel walls were aligned parallel to the simulation grid. Hence, no further interpolation procedures
are necessary to mimic the exact boundaries.
In order to test the accuracy of the VF and the IP method a test case with a curved geometry is necessary.
Furthermore it is possible to investigate the discretization error of the non-interpolating VP and DF method, with respect to curved geometries.

A simple adaption of the planar Poiseuille flow, is the laminar flow through a pipe, also referred to as Hagen-Poiseuille flow \citep{tritton88}.
The setup of the fluid domain is schematically shown in Fig. \ref{validation:setup_hpflow}.
It consist of a circular pipe with the radius $r_0$, which extends infinitely parallel to the $x$ axis.
This is realized similar to Sec. \ref{vali:sec_lpflow_setup} by using periodic boundaries in $x$-direction
and Immersed Boundary methods for the channel wall.
The center of the pipe is set to the center of the simulation domain ${(y_0, z_0) = (\nicefrac{l_y}{2}, \nicefrac{l_z}{2})}$.
In analogy to the Poiseuille flow, the velocity profile is a result from a predefined pressure gradient $\nicefrac{\partial p}{\partial x}$ in x-direction,
which  is added into the Navier-Stokes equation as an additional forcing term.

\begin{figure}[!bp]
      \centering
        \resizebox{0.75 \textwidth}{!}{
       \import{gfx/immersed_boundary/hpflow///}{setup.pdf_tex}
      }
      \caption{Setup for the Hagen-Poiseuille flow, consisting of a pipe of radius $r_0$ with the center at  ${(\nicefrac{l_y}{2}, \nicefrac{l_z}{2})}$ and of infinitesimal length.}
    \label{validation:setup_hpflow}
\end{figure}

An analytical solution of this problem is based of \citep{Kundu2012}.
Once again a steady state flow can be assumed, which implies that $\partial v_x/\partial t = 0$. With the introduction of cylindrical coordinates $(r, \phi, x)$
and the assumption that the flow is independent of $\phi$ the equation of motion reduces to

\begin{align}
    \label{vali:hpflow:navstok}
        0 &= - \frac{\partial p}{\partial x}  +  \frac{1}{\Rey} \frac{1}{r}\frac{\partial}{\partial r}\left(r\frac{\partial v_x}{\partial r}\right)
\end{align}

where $Re = \nicefrac{V_{m}r_0}{\nu}$.
The non-dimensional scales are similar to Sec. \ref{vali:sec_lpflow_setup}  except that the length scale is set to $r_0$ instead of $\Delta h$.

The integration of Eq. \ref{vali:hpflow:navstok} gives
\begin{align}
    v_x(r) &= \frac{r^2}{4}\frac{\partial p}{\partial x}\Rey + A \ln r + B.
\end{align}

It holds that $v_x(0)= V_{\text{M}}$ this condition can only be fulfilled if $A=0$.
With the use of the boundary conditions $v(r_0) = 0$ the velocity profile is given by
\begin{align}
    v_x(r) &= \frac{r^2 - r_0^2}{4}\frac{\partial p}{\partial x}Re.
\end{align}
Since $V_{M} \stackrel{!}{=} 1$ by definition (see Sec. \ref{vali:sec_lpflow_setup}),
 the pressure condition for the domain needs to be set to
\begin{align}
    \frac{\partial p}{\partial x} = -\frac{4 r_o^2}{Re}.
\end{align}

\subsubsection{Simulation: Grid Convergence Study}
%\paragraph{Simulation: Grid Convergence Study}\mbox{}\\

For an error evaluation a grid convergence study was performed with the Reynolds number set to $Re=100$.
The number of grid points was varied in the interval $N\in[32, 256]$. Furthermore, a
simulation with a resolution of $N=512$ was carried out.
Since the maximum velocity of the channel is $V_{M}=1$,
the sound speed was set to $c^2 = 100$ to fulfill the incompressibility condition $Ma = v/c < 0.1$.
The resulting time step for the highest resolution is $\Delta t = 10^{-4}$.
The main parameters of the simulation are  given by

\begin{center}
\vspace*{0.7ex}
\begin{tabular}{c|c|c|c|c|c|c|c }
 $ N  $                   & $\Delta t$ & $\Delta z, \Delta y$            & $\Rey$  & $c^2$   & $l_y, l_z, l_x$ & $r_0$ & $T_{end}$\\
\hline
 $512; [16, 256], \Delta N = 16 $& $10^{-4}$ & $\nicefrac{1}{N}$ & 100     & $100$   & 2.5, 2.5, $8\Delta y$ & 1     & 20\\
\end{tabular}.
\vspace*{0.7ex}
\end{center}

With this setup the VP, DF and IP methods were tested with FD2 and FD4 methods and optional with the VF method.
The IP method was additionally tested with the DF method. Thus, the velocity is interpolated on the
fluid boundary and set to zero outside of fluid domain.
For the VP method the non-dimensionalized damping was set to $J=10^{-4}$.
It would be possible to choose a larger time step for the lower resolution cases, which means that in general
one would apply a larger $J$ in a case of an application, due to the stability criterion $J>\Delta t$.
As a result the damping force on the boundaries $\propto \nicefrac{1}{J}$ would be smaller.
To remain consistent in the error convergence, here $\Delta t$ and
therefore $J$ are not altered.

\subsubsection{Simulation: Long-Term}
%\paragraph{Long-Term Simulations}\mbox{}\\

In order to test the  numerical stability and conservation of mass, a long-term simulation was performed.
A Reynolds number of $Re=100$ was chosen. The resolution is set to $N=96$ grid points and
the ending time was set to $T_{\text{end}}=1600$.
The main parameters of the simulation are  given by

\begin{center}
\vspace*{0.7ex}
\begin{tabular}{c|c|c|c|c|c|c|c }
 $ N  $                   & $\Delta t$ & $\Delta z, \Delta y$            & $\Rey$  & $c^2$   & $l_y, l_z, l_x $ &$r_0$ & $T_{end}$\\
\hline
 $96 $& $10^{-4}$ & $\nicefrac{1}{N}$ & 100     & $100$   & 2.5, 2.5, $8\Delta y $ & 1     & 1600\\
\end{tabular}.
\vspace*{0.7ex}
\end{center}

With this setup the same methods as in the grid convergence study were tested.

\subsection{Taylor-Couette Flow}

\begin{figure}[!bp]
  \begin{minipage}[c]{0.6\textwidth}
      \centering
        \resizebox{0.9 \textwidth}{!}{
       \import{gfx/immersed_boundary/tcflow//}{tcsystem.pdf_tex}
      }
  \end{minipage}
  \begin{minipage}[c]{0.3\textwidth}
      \caption{Setup of the Taylor-Couette flow, consisting of  two cylinders with radii $r_i$ and $r_o$ with angular velocities $\Omega_i$ and $\Omega_o$, that are oriented coaxial
       and parallel to the $z$ axis.
      \label{validation:setup_tcflow}
      }
  \end{minipage}
\end{figure}

The numerical setup of this test case is shown in Fig. \ref{validation:setup_tcflow}.
It contains two coaxial cylinders which are oriented parallel to the z-axis.
The inner cylinder has a radius and angular velocity $r_i$ and $\Omega_i$. In analogy these properties are defined as
$r_o$ and $\Omega_o$ for the outer cylinder.
The fluid domain is given by the gap of width $d = r_o - r_i$ between the cylinders and extends to infinite.
For the simulation  periodic boundaries are used in $z$-direction.

In contrast to the previously examined test cases this system provides new characteristics
of the flow regime at the fluid boundaries. The flow is not orthogonal to the curvature of the geometry and
Furthermore the velocities at the boundaries are not zero but have to fulfill  a Dirichlet condition given by
$\vec{v}(\vec{r})|_{r_i}  = \Omega_i \times \vec{r}$.

In dependency of the defined parameters different flow regimes exist.
For small differences in the rotation rates the flow is laminar and azimuthal.

An increase of the angular velocity $r_i$ above a critical value leads to an physical instability. The flow becomes unsteady.
In this flow regime toroidal vortices form, which are also denoted as Taylor cells.
The flow is not purely azimuthal \citep{tritton88}.
In this test case only the laminar and azimuthal flow regime will be considered, when the outer cylinder is not rotating ($\Omega_o = 0$).
In literature, this system is usually referred to as Circular Couette flow (CCF) \citep{Kundu2012}.

This problem can be reduced to two dimensions and be described in polar coordinates $(r, \phi)$. The equations of motion for the steady state are given by \citep{Kundu2012}
\begin{align}
    \label{vali:tc_flow_eqnavstok}
    -\frac{v^2_\phi}{r} = - \frac{\partial p}{\partial r} \qquad ;& \qquad 0 = \frac{1}{Re}\frac{\partial}{\partial r}\left(\frac{1}{r}\frac{\partial}{\partial r}(r v_\phi)\right).
\end{align}


For the non-dimensionalization the default convention is chosen according to  \citep{Chen2015}
\begin{align}
    \text{Length:}\qquad &  \vec{r} = \frac{\vec{r}^*}{r_o - r_i}  &,
    \qquad \text{Velocity:}\qquad& \vec{v} =  \frac{\vec{v}^*}{r_i\Omega_i},\\
    \text{Time:}  \qquad & t = t^* \cdot \frac{r_i \Omega_i}{r_o - r_i}&,
    \qquad  \text{Pressure:}\qquad & p = \frac{p^* - p_\infty}{\rho r_i^2\Omega_i^2}.
\end{align}


The flow is then characterized by the Reynolds number $Re = \nicefrac{\Omega_i R_i d}{\nu}$.
With the integration of Eq. \ref{vali:tc_flow_eqnavstok} the solution is given by (see [\citep{Kundu2012}])
\begin{align}
    v_\phi = Ar + \frac{B}{r}
\end{align}
where $A$ and $B$ are defined as
\begin{align}
    A := \frac{-\Omega_i r_i^2}{r^2_o - r^2_i} \qquad ,& \qquad B := \frac{\Omega_i r^2_i r^2_o}{r^2_o - r^2_i}.
\end{align}

\subsubsection{Simulations}

The simulations for this test case were carried out in analogy to the Hagen-Poiseuille flow, see Sec. \ref{vali:section_hpflow_start}.
Here the resolution is varied on the $x$ and $y$ axis, the radii are set to $r_i=1$ and $r_o=2$ and $l_x=l_y=2.5$.
The main parameters of the grid convergence study simulation are  given by

\begin{center}
\vspace*{0.7ex}
\begin{tabular}{c|c|c|c|c|c|c|c }
 $ N  $                   & $\Delta t$ & $\Delta x, \Delta y$            & $\Rey$  & $c^2$   & $l_x, l_y, l_z$ & $r_o, r_i$ & $T_{end}$\\
\hline
 $512, [16, 256], \Delta N = 16 $& $10^{-4}$ & $\nicefrac{1}{N}$ & 100     & $100$   & 2.5, 2.5, 8$\Delta x$ & 2, 1    & 20\\
\end{tabular}.
\vspace*{0.7ex}
\end{center}

The main parameters of the long-term simulations are  given by

\begin{center}
\vspace*{0.7ex}
\begin{tabular}{c|c|c|c|c|c|c|c }
 $ N  $                   & $\Delta t$ & $\Delta x, \Delta y$            & $\Rey$  & $c^2$   & $l_x, l_y, l_z$ &$r_o, r_i$ & $T_{end}$\\
\hline
 $96 $& $10^{-4}$ & $\nicefrac{1}{N}$ & 100     & $100$   & 2.5, 2.5, 8$\Delta x$   & 2, 1& 1600\\
\end{tabular}.
\vspace*{0.7ex}
\end{center}

