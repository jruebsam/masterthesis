\chapter{Conclusions and Outlook}

In the first part of this thesis different Immersed Boundary methods were
implemented and successfully tested on a GPU based algorithm.
The validation showed that these methods give good results for No-Slip boundaries,
the use with velocities different from zero at the boundaries tends to be problematic.
One concern of the validation is that only steady-state flows were used as test cases.
This approach is justified since the local error created trough approximations of the boundaries
has a larger influence on the global error .
However, for further validations unsteady flows could be of interest.
The validation results of the interpolation method showed by far the smallest numerical error.
Unfortunately this method became numerically unstable for the simulations of the librating cylinder and cone.
For possible future applications it would be important to find the origin of this instability.

The Immersed boundary methods introduced in this thesis were used to satisfy No-Slip boundaries.
In many scenarios it is necessary to use No-Flux and Free-Slip boundaries.
One example  is the Rayleigh-B\'{e}nard system.
In this system a fluid in a cubic or cylindric container is heated from below and cooled from above.
As a results the heat transport in this system is purely diffusive for small temperature gradients and
becomes convective for large temperature gradients above a critical instability.
For experiments and  numerical simulations of this system adiabatic side walls are often desired \citep{Lulff2011}.

A simple extension of the Volume-Penalization method was performed in \citep{Lulff2011}.
To enable No-Flux walls for the temperature, a inhomogenous thermal diffusivity $\kappa (x, y, z)$ was introduced.
By setting $\kappa(x, y, z) \ll 1$ in the boundary domain a suitable approximation of No-Flux walls can be obtained \citep{Lulff2011}.
Another improved version of the Volume-Penalization method for No-Flux walls can be found in \citep{Brown-Dymkoski2014}
in this method the No-Flux boundary is obtained by the forcing term
\begin{align}
    \vec{f}  = -\frac{H}{\nu_c}\left( n_k\pdn[v]{x_k} - q(\vec{r}, t) \right)
\end{align}
where $\nu_c$ is a regulation parameter, and $\vec{n}$ the normal to the boundary and $\vec{q}$ is the desired flux .
With this implementation the temperature flux is forced to zero in normal direction.
Both of these methods were tested in an implementation and are numerically stable.
However a proper validation need to be carried out in the future.
An implementation of Immersed boundary methods for No-Flux walls would give the possibility
to simulate this flow for example in a cylinder.

The use of Free-Slip boundaries could be useful to reduce the numerical oscillations in the simulation of the librating cone.
These oscillations are a results of the large Peclet number of the system and are generated
near the boundarie where the velocity gradients are the largest.
It could be observed that in simulation of precession driven dynamos in a cube with small Ekman numbers ($\propto 10^{-5}$),
no oscillations occur when using Free-Slip boudaries \footnote{Private communication O.Goepfert}.

A first implementation of Free-Slip boundaries was performed
by inserting diagonal walls into the fluid domain which overlap with grid points.
%The idea of this approach was to mirror the
On these walls the Free-Slip condition can simply be applied just like in the basic GPU algorithm,
using the the mirroring method introduced in Sec. ().
However a first test of this implemenation became numerically unstable which is not understood so far.
It would be desirable to obtain a stable  Free-Slip implementation for future applications.

There are some proposals regarding the basic implementation of the algorithm on the GPU.
The numerical oscillations in the density could be a concern for future simulations.
This problem seems to be quiet common for numerical computations involving first order derivates in CFD problems.
A common approach to avoid this would be the use of a staggered grid, where the velocities
are stored on the cell faces and the scalar fields at the cell centers.
Alternatively, different methods exists for pressure-based solvers where the pressure is interpolated to the cell faces.
One popular method in partical is the Rhie-Chow interpolaton \citep{uiae},
different improvements of this method can be found in literature \citep{uiae} that could be used for a future implementation.

It could be of further interest to implement an unstuctured cartesian grid into the GPU algorithm.
This is a really difficult task which would in return bring some drastically improvements
The error of the Immersed Boundary methods could be decreased by using a higher resolved grid in the vicinity to the boundarie.
This would furthermore the improve the resolution of boundary layers  and the oscillations
resulting from high Peclet numbers.



CONE:

- results short blabla
- possible experiment study
- wie weit ist der aufbau
- evtl piv etc
- turblent decay
- reprocduce the simulation s
