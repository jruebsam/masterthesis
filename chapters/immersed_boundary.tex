
\chapter{Immersed Boundary Methods for No-Slip Walls}
\section{Overview of Immersed Boundary Methods}

\begin{figure}[!bp]
  \centering
  \subfloat[cartesian grid]{\includegraphics[width=0.4\textwidth]{gfx/immersed_boundary/general_partition_triangle.jpg}\label{fig:grid_f0}}
  \hfil
  \subfloat[unstructured body-fitted grid]{\includegraphics[width=0.4\textwidth]{gfx/immersed_boundary/general_partition_triangle.jpg}\label{fig:grid_f2}}
  \caption{Different types of numerical grids}
\end{figure}

For many fluid problems it is mandatory to solve the equations of motion with respect to complex-shaped geometries \.
The algorithm introduced in section () is not yet suitable for such a scenario.
For instance the simulation inside a spheric geometry is impossible, since the boundaries
do not coincide with the implemented cartesian grid. Nevertheless there exist different approaches to overcome this problem,
which shall be introduced here. \\
The common approach to extend the algorithm would be to use a body-fitted mesh (see figure \ref{fig:grid_f1}),
different advantages and disadvantages arise with this kind of implementation (see \citep{Mittal2005}).
One benefit is a much simpler deployment of the desired boundary condition, due to the overlap of the grid with domain border.
Furthermore a higher accuracy can be achieved \citep{Gornak2013}.
However, using an unstructered grid generates plenty of computational overhead, during and before the execution of a simulation.
The generation of the grid is very complicated in contrast to using a cartesian grid, this can be even more complicated when
considering moving boundaries.
Also solving the finite differenc schemes on a curvilinear coordinate system, leads to more calculations on a single grid point.
The last important aspect is the implementation on the gpu.
Like discussed in section () it is more efficient to use homogenous storage and calculation pattern on a CUDA-device,
the use of unstructured data makes this very difficult.
Altough some attempts exists to solve these difficulties (see i.e. PAP), it is still uncertain if the obtained performance loss would be acceptable.\\
A set of alternative methods, to resolve the problems described above, are so called Immersed Boundary Methods.
The term was first mentioned in (PESKIN 1972), for the simulation of blood flow through a heartvale, but has since then been used for a variety of
methods (MITTAL).  All of them have the idea in common to perform the simulations on a cartesian grid which does not conform to the domain boundary.
To satisfy the desired boundary conditions additional terms are introduced into the equations of motion.
In general one can distinguish between contiuous forcing methods and direct forcing methods.
Continious forcing methods try to mimic the boundary using a localized force which acts on the boundary,
since the surface is tracked by lagrangian points this methods can be well suited for moving boundaries (MITTAL).
One common problem is that continous forcing can arise to stability problem and numerical oscillations in numericial stiff problem (SOURCE).
The direct forcing approach tries to satisfies the boundary condition, by imposing it directly to points near the fluid surface for example
trough an interpolaltion procedure.
Some of the major drawbacks using the IBM is the loss in  spatial accuracy at the boundary, therefore it can be necessary to use a higher grid resolution
compared to a body-fitted mesh.  Futhermore the non-conforming (?) boundaries are more difficult implement.
The benefits of these methods is the use of a cartesian grid, which is much more suited for a gpu-based implementation (see section X).
As a result the overall performance will probably be in the same order as the original algorithm.
In the thesis the Implementation of different Immersed Boundary Methods is seperated into three chapters depending on the boundary condition and application.
This chapter beginns with Implementation of NoSlip-Walls which are the easisest to implement.
The term Immersed Boundary Method is vaguely defined in literature, in this thesis we refer to it with all methods introduced in the following three chapters.


\section{Implemented IBMs}
\subsection{Volume Penalization}
Die Volume-Penalization Methode ermöglicht es, durch einen Kraftterm der auf die einzelnen Fluidzellen wirkt, mit wenig Aufwand Noslip-Ränder zu implementieren.
Das Verfahren wurde in mehreren Publikationen z.B. [bla] erfolgreich verwendet, eine mathematisch exaktere Abhandlung lässt sich z.B. in [bla2] finden.

\begin{wrapfigure}{r}{0.5\textwidth}
  \begin{center}
  \includegraphics[width=0.5\textwidth]{gfx/immersed_boundary/mask.png}\label{fig:mask_vp}
  \includegraphics[width=0.5\textwidth]{gfx/immersed_boundary/mask.png}\label{fig:mask_vp}
  \end{center}
  \caption{Maskierungsfunktion $H(x,y,z=const.) = x^2 + y^2 < c$ für einen Zylinder. }
\end{wrapfigure}

Das Volumen wird zunächst in einen Fluidbereich und einen festen Wandbereich, wie in Abb.1 dargestellt, unterteilt. Für die Differenzierung der Bereiche während der Simulation wird  eine Maskierungsfunktion
\begin{align}
H(x, y, z) = \begin{cases}
                    0, & \text{für } \vec{x}(x,y,z) \in Fluid, \\
                    1, & \text{sonst}.
             \end{cases}
\end{align}
verwendet. Als zusätzlicher Kraftterm wird nun eine exponentielle Dämpfung eingeführt die nur auf den Wandbereich des Volumens wirkt.
\begin{align}
\vec{f} = \frac{H(x, y, z)}{\nu}(\vec{v} - \vec{v_0})
\end{align}
Bei $\vec{v_0}$ handelt es sich um die gewünschte Randbedingung, der Kraftterm ist also proportional zur Auslenkung $\vec{v}$ eines Punktes vom gewünschten Ruhezustand.
Die Antwort des Kraftterms wird durch die Dämpfungrate $\nu$ reguliert. Je kleiner $\nu$ desto stärker ist die Dämpfungsrate, allerdings kann der Term
nicht beliebig klein gesetzt werden da die Stabilität für $\nu < dt$ nicht mehr gewährleistet ist [source].
Da für die Lösung der der Geschwindingskeitsfelder mit der Methode der künstliche Kompressibilität  bereits ein sehr kleiner Zeitschritt verwendet wird (s.Abb. X)
kann im Vergleich zu anderen Verfahren wie z.B. (pseudo-spektrale) eine relativ starke Dämpfungsrate verwendet werden.

\subsubsection{Validierung mit MASA}
-validierung mit masa für alle verfahren oben.. cube /evtl zylinder?
-vegl. und argumentation ränder ehh auf null.
-ein beispiel mit vol.pen.

\subsection{Direct Forcing}
Während die Volume Penalization Methode die Geschwindigkeit ausserhalb des Volumens nicht vollständig auf Null setzt,
 kann dies durch eine implizite Berechnung des Dämpfungsterm erreichtwerden. Es stellt sich heraus das dieser Ansatz equivalent
  zu der Direct Forcing Methode ist, die erstmals von [] verwendet und in [] beschrieben wird.
Betrachten wir zunächst den diskretisierten Zeitschritt
\begin{align}
    \frac{\vec{u}^{n+1} -\vec{u}^n}{\Delta t} = \mathscr{L} + \vec{f}\\
\end{align}
wobei $\mathscr{L}$ den diskretiesierten Operatoren der PDE entspricht.
Für einen Punkt auf dem Rand des Volumens soll nun die Randbedingung $\vec{u}^{n+1} = \vec{u}_0$ eingehalten werden.
Mit Formel () folgt
\begin{align}
    \frac{\vec{u}_0 -\vec{u}^n}{\Delta t} = \mathscr{L} + \vec{f} \Rightarrow \vec{f} = \frac{\vec{u}_0 -\vec{u}^n}{\Delta t\cdot \mathscr{L}}\\
\end{align}

Mit der Annahme dass der Rand mit dem numerischen Gitter übereinstimmt ist es nicht nötig den Kraftterm zur berechnen, stattdessen lässt sich der
Schritt vereinfachen in dem der Randwert nach  jedem Zeitschritt direkt auf die gewünschte Randbedingung gesetzt wird. Durch die
implizite Behandlung kommt es zu keiner weiter Stabilitätsbedingung.
Analog zur Volume-Penalization Methode wurde eine Serie von Simulationen für eine planare Poiseuille-Strömung durchgeführt.
Das Setup enstpricht dem gleichen wie in Abschnitt (X), lediglich das Verfahren wurde entsprechend angepasst und es wurden finite Differenzen Verfahren zweiter und
vierter Ordnung getestet.
In Abbildung () ist der relative Fehler, im Vergleich zur Volume Penalization  Methode mit $\nu=1e-4$ dargestellt.
Für das Verfahren vierter Ordnung liegt der Fehler im Bereich von 1\% und ist im Mittel doppelt so groß wie für die Volume-Penalization Methode.
Der Fehler für das Verfahren zweiter Ordnung verschwindet hingegen nahezu.
Das Verhalten lässt sich durch die Verwendung unterschliedlicher Schablonen, wie in Abb() dargestellt,  der finiten differenzen Verfahren erklären.
Während das Verfahren zweiter Ordnung nur den Randpunkt sieht, liegt bei der vierten Ordnung ein Punkt innerhalb des Randes.
Da auch dieser Wert auf Null gesetzt wird, kommt es zu einem fehler in der Berechnung von $\nabla u$.
\subsection{Volume fraction}
\subsection{Interpolation}
\newpage

\section{Validation}

In order to ensure a correct numerical behaviour of the introduced methods,
a large part of the thesis deals with the numerical validation.
Multiple examples from simple to more complex testcases are introduced in this section.
In general it is necessary to obtain a good evaluation of the numerical truncation error, the numerical stability over longer periods of time
and the fullfilment of the conversation laws, most importantly mass conservation.
Grid convergence studys against theoretical and high-resolution numerical solutions  will be performed
and compared for the different IBMs.


\subsection{Laminar Poiseuille-Flow}
\subsubsection{Theoretical description}

\begin{figure}[!bp]
  \centering
  \includegraphics[width=0.8\textwidth]{gfx/immersed_boundary/val_volpen/poiseuilleflow.png}\label{b}
  \caption{Theoretical setup of the poiseuille-flow channel.}
\end{figure}

%\begin{wrapfigure}{r}{0.5\textwidth}
%  \begin{center}
%      \includegraphics[width=0.48\textwidth]{gfx/immersed_boundary_methods/val_volpen/poiseuilleflow.png}
%   \end{center}
%          \caption{ paxvleb xvle}
%\end{wrapfigure}


The first testcases is the laminar poiseuille-flow, the theoretical setup is presented in figure ().
It consist of two infintiy long planes at $z=h_1$ and $z=h_2$, which are oriented parallel to the xy-plane and the distance $\Delta h = h_2 - h_1$ between them.
Numerically this is realized by using periodic boundaries in xy-direction and no-slip boundaries in z-direction.
The velocity profile results from an external force, given by a  pressure gradient in x-direction, which is introduced into the naviers-stokes equation.
No other forces are present furthermore the flow will be independent of the y coordinate,
hence for the steady state $\partial v_x /\partial t = 0$ the equations of motion can be reduced to:

\begin{align}
\frac{\partial v_x}{\partial t} &= - \frac{\partial p}{\partial x} + \nu \frac{\partial^2 v_x}{\partial z^2} = 0
\end{align}

The equation can simply be integrated twice which yields the solution

\begin{align}
v_x &= \frac{1}{2\nu}\frac{\partial p}{\partial x}z^2 + zc_1 + c_2
\end{align}

Using the noslip-boundary condition $v_x(h_1) = v_x(h_2) = 0$ and furthermore by defining
$A:=\frac{1}{2\nu}\frac{\partial p}{\partial x}$ one obtains the additional conditions

\begin{align}
c_1 &= A\frac{h_1^2 -h_2^2}{h_2 - h_1} = -A(h_1+h_2)\\
c_2 &= A(h_1(h_1 + h_2) - h_1^2) = Ah_1h_2\\
\end{align}

The velocity is than given by the quadratic function

\begin{align}
v_x &= A(z^2 - z(h_1 + h_2) + h_1h_2)
\end{align}

The maximum velocity and postinon can be obtained by simple calculus

\begin{align}
z_{max} &= \frac{h_1+h_2}{2} \wedge v_{max} = A\left(h_1h_2 - \frac{(h_1 + h_2)^2}{4}\right)
\end{align}

and the reynolds number of the system is given by
\begin{align}
    Re &= \frac{v_{max} }{\nu} = \frac{1}{2\nu^2}\pdn[p]{x}\left(h_1h_2 - \frac{(h_1 + h_2)^2}{4}\right)
\end{align}



With the given theoretical solution, the next objective is the comparison
to the default implementation, the volume penalization method and the direct forcing method.
Since we have a flow parallel to the grid  it does not make sence to compare it to the interpolation methods
For the comparision with a theoretical solution it is necessary to ensure that the surface grid points match with the total height $h$ of the channel.
In the default setup the noslip-boundaries are realized with the default implementation, like explained in section ().
For the immersed boundary methdos the upper- and lower boundaries are given by the masking function.
\begin{align}
H(x, y, z) = \begin{cases}
                    0, & \text{for \  }  z < h_2 \lor z>h_1 \\
                    1, & \text{else}.
             \end{cases}
\end{align}

\subsubsection{Test of the Default Implementation}

\begin{figure}[!bp]
    \centering
    \includegraphics{gfx/immersed_boundary/poiseuille_flow/1_default/relative_l2error.pdf}
    \caption{Relative $l_2$-error for second and fourth order schemes of the default algorithm.\label{fig:ema1}}
\end{figure}

Initially a grid convergence test was performed with the default implementation of the algorithm, without the use of immersed boundaries.
For this test case this is still possible since the geometry is non-curved and parallel to the cartesian grid.\\
For the  numerical setup the parameters were set to $l_x=1$, $l_y=0.25$, $l_z=1$.
The reynolds number was set to a constant value of $Re=500$, whereas the resolution $N$ was varied in the Intervall $[16 - 128]$ with a stepsize of $\Delta N = 8$.
The timestep was set to $\mathrm{dt}=1e-5$ and the pressure gradient to $\partial p \partial x  = 10$.
For all resolutions the simulations were performed for finite differcen schemes of second and fourth order.
The results are shown in figure \ref{fig:ema1} on a double logarithmic plot.\\
For the finite difference scheme of second order, the error behaves not as assumed.
Instead of the decline one would expect with the decrease of the resolution, the  error is nearly constant.
This behaviour can be explained due to the lack of complexity of the test case. As the theoretical solution is a polynom of second order
no higher order terms will occur in the numerical solution, hence the second order scheme is capable of a perfect approximation, independent of the
grid resolution. The remaining error terms, are of the order $10e^{-8}$, which is extremly small and occurs due to the floating point roundoff.
With this in mind the behavior of the fourth-order scheme is even more unexpected.
We see a linear decrease of the error on the log scale.
The result of a power/poly?  fit yields a convergence rate of order $\approx 2$.
Furthermore the error is much larger compared to the second order scheme, between $10e^{-2}$ and $10e^{-5}$.
The test case reveals that an error exists in the default implementation of the boundarie conditions.
An explanation can be given with comparison of the theoretical solution. The laplace operator is given by
 $ \nu \pdn[^2 v_x]{x^2} = 2A = \frac{1}{\nu}\pdn[p]{x}$
Using the mirroring method creates a discontinuous function at the boundaries, since the laplace operator changes the sign.
When using the second order method the tree-point-stencil evaluates to the correct value $\pdn[^2 v_x]{x^2} = 1$.
The five-point-stencil of the fourth orderer scheme evaluates to $\pdn[^2 v_x]{x^2} = 1$, since it uses one point behind the boundary.
As a result the discontinuity creates an error in the higher order scheme.

However we have to keep in mind that this test case is strongly dependent of the boundarie condition,
which is responsible for damping the flow on the sides of the channel and therefore for the overall velocity profile.
In general we can't say that the fourth order scheme is inferior, it depends how large of an impact
the boundaries have to the fluid problem.
For a problem where teh boundarie is not htat import  blabla.
We have to keep in mind that the error for an acceptable resolution is around (), which is still small(how small?).
One possibility to avoid this error, would be to use an asymmetrich stencil, for example see [].
This could also lead to computational overhead due to gpu blabla.



\begin{figure}[!hbtp]
  \centering
  \includegraphics{gfx/immersed_boundary/poiseuille_flow/2_vp/vp_profile.pdf}\label{fig:vp_flow}
  \caption{Geschwindigkeistprofile im Kanal bei Variation der Dämpfungskonstante $\nu$ und Reynoldszahl $Re=500$.}
\end{figure}

\subsubsection{Test of the Volume Penalization Method}

The next objective is to investigate the error of the volume penalization method.
We begin by studying the behavior of the velocity profile, with variation of the Reynolds-number and the damping parameter $\nu$, in the regime $Re=100-500$ and $\nu=100-100$.
The numerical setup is equal to the one for the default methods, except we now introduce the masking function (LABELABOVE) with $h_1=0.25$ and $h_2=0.75$.
To make sure that the channel width is equal to $\Delta h = 1$, the total height was set to $l_z\approx2.01587$. This furthermore ensures that the grid points overlap exactly
with the masking function at $h_1$ and $h_2$. The resolution was set to $N_x\times N_y\times N_z = 64\times16\times128$.\\
The resulting velocity profile is exemplarily shown in figure \ref{fig:vp_flow}, for varying $\nu$ and a constant reynolds number $Re=500$.\\
It can be noted that with an decrease of $\nu$, the numerical solution converges against the theoretical one.
The quadratic part of the velocity profile inside the fluid domain is independent of the damping constant $\nu$.
Since the damping can not fullfill the exact boundarie conditions, a slight offset is introduced into the velocity profile
which leads to an offset in the solution.
In the masked area of the volume we can see an exponential decrease of the total velocity.
REWRITE THIS This is logical result since the penalization term, leads to an exponential damped solution like discussed in section ().\\
For an error estimation, the absolute and relative $l_2$-error were computed inside the fluid domain,
The results are shown in figure \ref{fig:vp_error}.

\begin{figure}[!bpt]
  \centering
  \includegraphics{gfx/immersed_boundary/poiseuille_flow/2_vp/vp_error.pdf}\label{fig:vp_error}
  \caption{Absoluter und relativer Fehler in Abhängigkeit von Dämpfungskonsante $\nu$ und Reynoldszahl $Re$.}
\end{figure}

- for both error we can see a decrease until $\nu=1e-4$, followed by a slight increase as one can see in the inset
- what kind of decrease ?
- the behaviour will be clarified in the following grid convergence study
- the absolute error is growing with a decrease in the reynoldsnumber for $\nu>5\cdot10^{-3}$.
- this is as expected since the stress on the wall is larger, in the area




Betrachtet man den absoluten Fehler, so fällt auf das im Bereich , mit steigender Reynoldszahl, der Fehler zunimmt.
Dies entspricht zunächst der Erwartung, da das Geschwindigkeitsprofil mit steigender Reynoldszahl stärker an der Wand zerrt.
Allerdings kommt es im Bereich $\nu \in [1e-3 - 1e-2]$ zu einem umgekehrten Verhalten, der Fehler nimmt mit der Reynoldszahl ab,
Du Ursache hierfür liess sich nicht eindeutig klären.
Für den relativen Fehler lässt sich ein Abfall mit steigender Reynoldszahl beobachten. Da $\vec{f} \propto (\vec{v}-\vec{v_0})  \propto Re$ wird der Reibunsterm
proportional zur Reynoldszahl skaliert. Der Fehler durch den Rand ändert sich, im Vergleich zum Geschwindigskeitsprofil kaum, wodurch der Abfall zustande kommt.
Der relative Fehler fällt bei $\nu=1e-4=10\Delta t$ auf unter 1\%, wodurch dieser Wert als geeignet angesehen werden kann für zukünftige Simulationen mit der Volume-Penalization Methode.
-todo: fluktuation im rand bei vp?

-grid convergence study o2 and o4
\begin{figure}[!bpt]
  \centering
  \includegraphics{gfx/immersed_boundary/poiseuille_flow/2_vp/vp_convergence.pdf}\label{fig:vp_conv}
  \caption{Absoluter und relativer Fehler in Abhängigkeit von Dämpfungskonsante $\nu$ und Reynoldszahl $Re$.}
\end{figure}


\subsection{Direct Forcing Methode}

