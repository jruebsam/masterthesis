
\chapter{Immersed Boundary Methoden}

\begin{figure}[!bp]
  \centering
  \subfloat[kartesisches Gitter]{\includegraphics[width=0.4\textwidth]{gfx/immersed_boundary_methods/general_partition_triangle.jpg}\label{fig:f1}}
  \hfill
  \subfloat[unstrukturiertes Gitter]{\includegraphics[width=0.4\textwidth]{gfx/immersed_boundary_methods/general_partition_triangle.jpg}\label{fig:f2}}
  \caption{Beispiele für numerische Gitter}
\end{figure}

\section{Einleitung}
Die bisher eingeführten Methoden eignen sich um auf einfachen Geometrien Simulationen durchzuführen.
Wenn die Ränder des Simulations-Gebietes nicht mehr mit dem numerischen Gitter übereinstimmen, lassen
sich die in Abschnitt () eingeführten Randbedingungen nicht mehr verwenden.
Das Problem lässt sich durch die Einführung eines an den Rand angepassten Gitter umgehen (s.Abb. \label{fig:f1}),
führt allerdings zu einem deutlich höheren Aufwand in den numerischen Berechnungen.
Die Berechnung des Gitters kann sehr aufwendig werden und es ist nicht eindeutig klar unter welchen
Performance-Einbussen sich das neue Gitter in der GPGPU-Implementierung verwenden lässt, da Bedingungen wie
z.B. Coalesced Readings  (s.Abbschnitt x) durch die unstrukturierten Daten deulich erschwert wären.
Eine Alternative die in dieser Arbeit verwendet wird sind  sog. Immersed Boundary Methoden.


- Beschreibung allgemein verschiedene immersed boundary methoden direkt /exp/ imp etc
    -erstmal peskin et al.
    -im prinzip 3 verschiedene ansätze continuous / direct /  ghost cell methoden
    - hier einfache methoden wegen gpu


\newpage

\section{No-Slip-Boundaries}
Für die unterschiedlichen Skalare in den Bousinnesqe gleichungen gibt es eine Vielzahl an möglichen Randbedingungen (s.Abbschnitt. X).
Während für die Geschwindigkeitsfelder häufig No- oder Free-Slip Ränder verwendet werden,
kommen bei der Temperatur oft NoFlux-Randbedingungen zum Einsatz um einen Wärmeaustausch mit der Umgebung zu unterdrücken.
Wie sich heraussgestellt hat ist die Implementierung von NoFlux-wänden mittels IBM wesentlich komplexer.
Die einzelnen Randbedingungen wurde daher getrennt voneinander untersucht, zunächst werden NoSlip-Ränder (s. gl. X) betrachtet.

\subsection{Volume Penalization}
Die Volume-Penalization Methode ermöglicht es, durch einen Kraftterm der auf die einzelnen Fluidzellen wirkt, mit wenig Aufwand Noslip-Ränder zu implementieren.
Das Verfahren wurde in mehreren Publikationen z.B. [bla] erfolgreich verwendet, eine mathematisch exaktere Abhandlung lässt sich z.B. in [bla2] finden.

\begin{wrapfigure}{r}{0.5\textwidth}
  \begin{center}
  \includegraphics[width=0.5\textwidth]{gfx/immersed_boundary_methods/mask.png}\label{fig:mask_vp}
  \includegraphics[width=0.5\textwidth]{gfx/immersed_boundary_methods/mask.png}\label{fig:mask_vp}
  \end{center}
  \caption{Maskierungsfunktion $H(x,y,z=const.) = x^2 + y^2 < c$ für einen Zylinder. }
\end{wrapfigure}

Das Volumen wird zunächst in einen Fluidbereich und einen festen Wandbereich, wie in Abb.1 dargestellt, unterteilt. Für die Differenzierung der Bereiche während der Simulation wird  eine Maskierungsfunktion
\begin{align}
H(x, y, z) = \begin{cases}
                    0, & \text{für } \vec{x}(x,y,z) \in Fluid, \\
                    1, & \text{sonst}.
             \end{cases}
\end{align}
verwendet. Als zusätzlicher Kraftterm wird nun eine exponentielle Dämpfung eingeführt die nur auf den Wandbereich des Volumens wirkt.
\begin{align}
\vec{f} = \frac{H(x, y, z)}{\nu}(\vec{v} - \vec{v_0})
\end{align}
Bei $\vec{v_0}$ handelt es sich um die gewünschte Randbedingung, der Kraftterm ist also proportional zur Auslenkung $\vec{v}$ eines Punktes vom gewünschten Ruhezustand.
Die Antwort des Kraftterms wird durch die Dämpfungrate $\nu$ reguliert. Je kleiner $\nu$ desto stärker ist die Dämpfungsrate, allerdings kann der Term
nicht beliebig klein gesetzt werden da die Stabilität für $\nu < dt$ nicht mehr gewährleistet ist [source].
Da für die Lösung der der Geschwindingskeitsfelder mit der Methode der künstliche Kompressibilität  bereits ein sehr kleiner Zeitschritt verwendet wird (s.Abb. X)
kann im Vergleich zu anderen Verfahren wie z.B. (pseudo-spektrale) eine relativ starke Dämpfungsrate verwendet werden.

\subsubsection{Validierung mit MASA}
-validierung mit masa für alle verfahren oben.. cube /evtl zylinder?
-vegl. und argumentation ränder ehh auf null.
-ein beispiel mit vol.pen.

\subsubsection{Validierung : planare Poiseuille Strömung}
Es stellt sich die Frage in welcher Größenordnung die Dämpfungskonstante $\nu$ der Volume Penalization methode liegen muss, um einen möglichst kleinen
Fehler zu gewährleisten. Ein einfacher Testfall der sich hierfür betrachen lässt ist eine einfache planare poiseuille strömung, diese ist schematisch in Abb. (x). dargestellt.
\paragraph*{Theoretische Beschreibung}\mbox{}\\
Wir betrachten eine laminare Strömung in x-Richtung die durch einen Druckgefälle $f=-\frac{\partial p}{\partial x}$ angetrieben wird.
Für die x- und y-Richtung werden periodische Randbedingungen angenommen. In z-Richtung wird das Volumen durch zwei Ebenen bei $h_1$ und $h_2$ begrenzt,
es gilt $ \vec{v}(z=h_1) = \vec{v}(z=h_2) = 0$.
Im stationären Fall lässt sich  die Bewegungsgleichung dann  auf eine Dimension reduzieren, es gilt:

\begin{align}
\frac{\partial v_x}{\partial t} &= - \frac{\partial p}{\partial x} + D \frac{\partial^2 v_x}{\partial z^2} = 0 \\
\Rightarrow v_x &= \frac{1}{2D}\frac{\partial p}{\partial x}z^2 + zc_1 + c_2\\
\end{align}

Mit $\vec{v}(h_1) = \vec{v}(h_2) = 0$ und $A:=\frac{1}{2D}\frac{\partial p}{\partial x}$ ergibt sich:
\begin{align}
c_1 &= A\frac{h_1^2 -h_2^2}{h_2 - h_1} = -A(h_1+h_2)\\
c_2 &= A(h_1(h_1 + h_2) - h_1^2) = Ah_1h_2\\
\Rightarrow v_x &= A(z^2 - z(h_1 + h_2) + h_1h_2)
\end{align}
Da die Strömung in der Kanalmitte am stärksten ist gilt zudem:
\begin{align}
z_{max} &= \frac{h_1+h_2}{2}\Rightarrow v_{max} = A\left(h_1h_2 - \frac{(h_1 + h_2)^2}{4}\right)
\end{align}
Für die Strömung lässt sich die Reynoldszahl dann gemäß $Re \propto \frac{v_{max}}{D}$  bestimmen.
\begin{figure}[!hbtp]
  \centering
  \includegraphics[width=0.9\textwidth]{gfx/immersed_boundary_methods/vp_flow.png}\label{fig:vp_flow}
  \caption{Geschwindigkeistprofile im Kanal bei Variation der Dämpfungskonstante $\nu$ und Reynoldszahl $Re=500$.}
\end{figure}

\paragraph*{Setup}\mbox{}\\
Um die Abhängigkeit des Fehlers von der Dämpfungskonstante zu betrachten wurde ein Kanal mit $l_x=1$, $l_y=1$ und $l_z=2$ sowie $h_1=0.25$, $h_2=0.75$.
betrachtet. Die Maskierungsfunction ergibt sich damit gemäß $H(z) = (z>h1) \wedge (z<h2)$.
Für die Reynoldszahl wurden Werte im Intervall $Re \in [100, 500]$ verwendet, die Dämpfungskonstante wurde zwischen $\nu \in [1e-5, 0.1]$ variert, während der Zeitschritt mit $dt =1e-5$ konstant gehalten wurde. Die genauen Angaben für alle Parameter sind in (Anhang Tab.X) zu finden.

\paragraph*{Ergebnisse}\mbox{}\\
Zunächst ist in Abb. \ref{fig:vp_flow} das Geschwindigkeitsprofil der Strömung für $Re=500$ exemplarisch dargstellt. Es lässt sich bereits qualitativ sehr gut erkennen, dass für eine
starke Dämfung die Kanalströmung an den Grenzen $h_1$ und $h_2$ verschwindet und gut mit der dem theoretischen Profil übereinstimmt.
Bei einem Verringern der Dämpfungrate entwickelt sich im Rand ein Geschwindigkeitsprofil.
Um sicherzustellen dass sich das Profil vollständig entwickelt wurde die Simulation bis zu dem Zeitpunkt fortgegeführt
 in welchem die kinetische Energie des Systems einen stationären Wert erreicht. Anschließend wurde der absolute und relative Fehler mit Formel (X) berechnet,
dabei wurde das theoretische Profil gemäß () verwendet. Die Ergebnisse sind in Abbildung \ref{fig:vp_error} dargestellt.

\begin{figure}[!bp]
  \centering
  \includegraphics[width=1.0\textwidth]{gfx/immersed_boundary_methods/vp_error.png}\label{fig:vp_error}
  \caption{Absoluter und relativer Fehler in Abhängigkeit von Dämpfungskonsante $\nu$ und Reynoldszahl $Re$.}
\end{figure}

Bei beiden Fehler lässt sich ein Abfall bis $\nu=1e-4$ erkennen, anschließend kommt es zu einem minimalen Anstieg.
Betrachtet man den absoluten Fehler, so fällt auf das im Bereich $\nu>5\cdot10^{-3}$, mit steigender Reynoldszahl, der Fehler zunimmt.
Dies entspricht zunächst der Erwartung, da das Geschwindigkeitsprofil mit steigender Reynoldszahl stärker an der Wand zerrt.
Allerdings kommt es im Bereich $\nu \in [1e-3 - 1e-2]$ zu einem umgekehrten Verhalten, der Fehler nimmt mit der Reynoldszahl ab,
Du Ursache hierfür liess sich nicht eindeutig klären.
Für den relativen Fehler lässt sich ein Abfall mit steigender Reynoldszahl beobachten. Da $\vec{f} \propto (\vec{v}-\vec{v_0})  \propto Re$ wird der Reibunsterm
proportional zur Reynoldszahl skaliert. Der Fehler durch den Rand ändert sich, im Vergleich zum Geschwindigskeitsprofil kaum, wodurch der Abfall zustande kommt.
Der relative Fehler fällt bei $\nu=1e-4=10\Delta t$ auf unter 1\%, wodurch dieser Wert als geeignet angesehen werden kann für zukünftige Simulationen mit der Volume-Penalization Methode.

-todo: fluktuation im rand


\subsection{Direct Forcing Methode}
Während die Volume Penalization Methode die Geschwindigkeit ausserhalb des Volumens nicht vollständig auf Null setzt,
 kann dies durch eine implizite Berechnung des Dämpfungsterm erreichtwerden. Es stellt sich heraus das dieser Ansatz equivalent
  zu der Direct Forcing Methode ist, die erstmals von [] verwendet und in [] beschrieben wird.
Betrachten wir zunächst den diskretisierten Zeitschritt
\begin{align}
    \frac{\vec{u}^{n+1} -\vec{u}^n}{\Delta t} = \mathscr{L} + \vec{f}\\
\end{align}
wobei $\mathscr{L}$ den diskretiesierten Operatoren der PDE entspricht.
Für einen Punkt auf dem Rand des Volumens soll nun die Randbedingung $\vec{u}^{n+1} = \vec{u}_0$ eingehalten werden.
Mit Formel () folgt
\begin{align}
    \frac{\vec{u}_0 -\vec{u}^n}{\Delta t} = \mathscr{L} + \vec{f} \Rightarrow \vec{f} = \frac{\vec{u}_0 -\vec{u}^n}{\Delta t\cdot \mathscr{L}}\\
\end{align}
Mit der Annahme dass der Rand mit dem numerischen Gitter übereinstimmt ist es nicht nötig den Kraftterm zur berechnen, stattdessen lässt sich der
Schritt vereinfachen in dem der Randwert nach  jedem Zeitschritt direkt auf die gewünschte Randbedingung gesetzt wird. Durch die
implizite Behandlung kommt es zu keiner weiter Stabilitätsbedingung.
Analog zur Volume-Penalization Methode wurde eine Serie von Simulationen für eine planare Poiseuille-Strömung durchgeführt.
Das Setup enstpricht dem gleichen wie in Abschnitt (X), lediglich das Verfahren wurde entsprechend angepasst und es wurden finite Differenzen Verfahren zweiter und
vierter Ordnung getestet.
In Abbildung () ist der relative Fehler, im Vergleich zur Volume Penalization  Methode mit $\nu=1e-4$ dargestellt.
Für das Verfahren vierter Ordnung liegt der Fehler im Bereich von 1\% und ist im Mittel doppelt so groß wie für die Volume-Penalization Methode.
Der Fehler für das Verfahren zweiter Ordnung verschwindet hingegen nahezu.
Das Verhalten lässt sich durch die Verwendung unterschliedlicher Schablonen, wie in Abb() dargestellt,  der finiten differenzen Verfahren erklären.
Während das Verfahren zweiter Ordnung nur den Randpunkt sieht, liegt bei der vierten Ordnung ein Punkt innerhalb des Randes.
Da auch dieser Wert auf Null gesetzt wird, kommt es zu einem fehler in der Berechnung von $\nabla u$.

\begin{figure}[!tpb]
  \centering
  \includegraphics[width=0.6\textwidth]{gfx/immersed_boundary_methods/dfo2o4.png}\label{fig:df_o2o4}
  \caption{bla}
\end{figure}

-Zeitserie

-einleitung gekrümmte geometrien

\subsection{Direct Forcing mit Volume Fraction}
-paper quote formeln
-implementierung beispiel

\section{Direct Forcing mit Interpolation}
-paper quote formeln
-implementierung beispiel

\section{Methoden-Vergleich und numerische Validierung}
In diesem Abschnitt sollen die verschiedenen Methoden über numerische Testfälle validiert und miteinnander
verglichen werden.


\subsubsection{Poiseuille Strömung im Zylinder}

\subsubsection{Zusammenfassung}


\subsection{No-Flux-Boundaries}

\subsubsection{'Variable Konduktivität'}










